\chapter*{Donders Graduate School for Cognitive Neuroscience}
\label{ch:dgn}
\chaptermark{Donders Graduate School}
\addcontentsline{toc}{chapter}{Donders Graduate School for Cognitive Neuroscience}

For a successful research Institute, it is vital to train the next generation of scientists. To achieve this goal, the Donders Institute for Brain, Cognition and Behaviour established the Donders Graduate School in 2009. The mission of the Donders Graduate School is to guide our graduates to become skilled academics who are equipped for a wide range of professions. To achieve this, we do our utmost to ensure that our PhD candidates receive support and supervision of the highest quality.

Since 2009, the Donders Graduate School has grown into a vibrant community of highly talented national and international PhD candidates, with over 500 PhD candidates enrolled. Their backgrounds cover a wide range of disciplines, from physics to psychology, medicine to psycholinguistics, and biology to artificial intelligence. Similarly, their interdisciplinary research covers genetic, molecular, and cellular processes at one end and computational, system-level neuroscience with cognitive and behavioural analysis at the other end. We ask all PhD candidates within the Donders Graduate School to publish their PhD thesis in de Donders Thesis Series. This series currently includes over 600 PhD theses from our PhD graduates and thereby provides a comprehensive overview of the diverse types of research performed at the Donders Institute. A complete overview of the Donders Thesis Series can be found on our website: \url{https://www.ru.nl/donders/donders-series}

The Donders Graduate School tracks the careers of our PhD graduates carefully. In general, the PhD graduates end up at high-quality positions in different sectors, for a complete overview see \url{https://www.ru.nl/donders/destination-our-former-phd}. A large proportion of our PhD alumni continue in academia (>50\%). Most of them first work as a postdoc before growing into more senior research positions. They work at top institutes worldwide, such as University of Oxford, University of Cambridge, Stanford University, Princeton University, UCL London, MPI Leipzig, Karolinska Institute, UC Berkeley, EPFL Lausanne, and many others. In addition, a large group of PhD graduates continue in clinical positions, sometimes combining it with academic research. Clinical positions can be divided into medical doctors, for instance, in genetics, geriatrics, psychiatry, or neurology, and in psychologists, for instance as healthcare psychologist, clinical neuropsychologist, or clinical psychologist. Furthermore, there are PhD graduates who continue to work as researchers outside academia, for instance at non-profit or government organizations, or in pharmaceutical companies. There are also PhD graduates who work in education, such as teachers in high school, or as lecturers in higher education. Others continue in a wide range of positions, such as policy advisors, project managers, consultants, data scientists, web- or software developers, business owners, regulatory affairs specialists, engineers, managers, or IT architects. As such, the career paths of Donders PhD graduates span a broad range of sectors and professions, but the common factor is that they almost all have become successful professionals.

For more information on the Donders Graduate School, as well as past and upcoming defences please visit: \url{http://www.ru.nl/donders/graduate-school/phd/}