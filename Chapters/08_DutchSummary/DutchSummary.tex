\chapter*{Dutch Summary}
\label{ch:dutchsummary}
\chaptermark{Dutch Summary}
\addcontentsline{toc}{chapter}{Dutch Summary}


% intro
Om een goed gesprek te kunnen voeren, zijn in ieder geval twee vaardigheden noodzakelijk: het begrijpen van woorden en zinnen en het bedenken wat andere mensen denken en geloven. Die laatste vaardigheid wordt ook wel \emph{mentalizing} genoemd. Van beide vaardigheden is bekend dat ze minder goed ontwikkeld kunnen zijn in mensen met Autisme Spectrum Stoornis. Autistische mensen ervaren regelmatig problemen bij het houden van gesprekken in verbale en non-verbale communicatie. Autistische mensen lopen ook vaak achter in hun taalontwikkeling in vergelijking met niet-autistische mensen. Voor een lange tijd leek psychologisch onderzoek aan te tonen dat autistische mensen minder goed zijn in mentalizing dan niet-autistische mensen. Er zijn echter een aantal dingen aan te merken op de manier waarop veel van dit onderzoek is uitgevoerd. Veel experimenten zijn erg gekunsteld op een manier dat ze niet goed weergeven hoe mensen in de echte wereld nadenken over de gedachten en overtuigingen van een ander. Ook kan bij sommige experimenten het verminderde mentalizingniveau van autistische mensen verklaard worden door een andere factor, zoals taalvaardigheid van de proefpersonen. Bovendien tonen meer recente experimenten aan dat autistische mensen mentalizingtaken even goed volbrengen als niet-autistische mensen. Het was dus nog onduidelijk hoe deze twee groepen verschillen in mentalizing als dit onderzocht wordt in een meer re\"eele setting.

Een andere conditie waarbij mensen problemen ervaren tijdens communicatie is sociale angst, gekenmerkt door een ernstige angst om negatief beoordeeld te worden door anderen. Een gevolg hiervan kan zijn dat sociaal angstige mensen sociale situaties helemaal vermijden en zich isoleren. Een mogelijke verklaring voor deze angst is een verminderd mentalizingvermogen, wat leidt tot stress bij het proberen te interpreteren van het gedrag van anderen. Een andere verklaring is dat sociale cues en informatie anders binnenkomen bij sociaal angstige mensen en als het ware gemonitord worden voor mogelijke sociale dreigingen. Bij het testen van deze verklaringen stuiten we op dezelfde problemen als het onderzoek naar mentalizing in autisme. Daar bovenop komt ook nog dat sociaal angstige proefpersonen veel prestatiedruk kunnen ervaren tijdens een experiment waar ze telkens opdrachten krijgen om te voltooien. Het is dus belangrijk dat sociaal angstige proefpersonen tijdens een taak zo min mogelijk het gevoel hebben dat ze beoordeeld worden.

Om de communicatieproblemen in autistische en sociaal angstige mensen volledig te kunnen begrijpen en op zoek te gaan naar behandelingen is het van belang dat we kijken of er een biologische verklaring is voor hun klachten. Daarom hebben we in het onderzoek in dit proefschrift vooral gekeken naar de signalen in de hersenen terwijl autistische, sociaal angstige, en neurotypische (geen van beide condities) mensen taken deden zoals zinnen lezen of een video kijken waarin personages met elkaar communiceren. Op deze manier kunnen we aanknopingspunten vinden voor behandelingen of therapie die hun dagelijkse problemen kunnen verminderen. 

% chapter 2
In hoofdstuk~\ref{ch:mentalizing_asc} hebben we autistische en neurotypische mensen een video laten kijken waarin twee bevriende personages op een probleem stuiten en dat proberen op te lossen. Belangrijk is dat de video geen taal bevat, wat betekent dat verschillen in taalvaardigheid de resultaten niet kunnen be\"invloeden. Dit doen ze terwijl ze in een MRI-scanner (Magnetic Resonance Imaging) scanner liggen om hun hersensignalen te meten. Ook wordt hun pupilgrootte gemeten, wat een indicatie geeft hoeveel inspanning het vergt om een bepaald beeld of geluid te waarnemen en interpreteren. Na het kijken van de video beantwoordden ze een aantal vragen over de video. Als we de hersensignalen, pupilgroottes en beschrijvingen van de twee groepen vergelijken, zien we dat er geen verschillen zitten tussen de autistische en neurotypische proefpersonen in de mate waarin ze bedenken wat de gedachten van de personages inhouden. Wel zien we dat autistische proefpersonen over de hele video onderling meer wisselende hersenactiviteit laten zien, in vergelijking met de meer onderling vergelijkbare neurotypische proefpersonen. Dit ligt niet in lijn met andere, vergelijkbare studies. We denken dat dit mogelijk kan verklaard worden door eigenschappen van de specifieke video die gebruikt wordt in dit type experiment.

% chapter 3
Voor het onderzoeken van mentalizing en het interpreteren van sociale interacties in sociale angst hebben we in hoofdstuk~\ref{ch:mentalizing_sa} een vergelijkbare aanpak gebruikt als in hoofdstuk~\ref{ch:mentalizing_asc}. Sociaal angstige en niet sociaal angstige mensen keken dezelfde video als in hoofdstuk~\ref{ch:mentalizing_sa} in een MRI-scanner, waarbij we weer ge\"interesseerd waren in de reactie van de hersenen, de pupilgrootte en achteraf de beschrijvingen van de video. Cruciaal is dat de proefpersonen geen instructie kregen om vragen te beantwoorden om prestatiedruk te minimaliseren. In de data zagen we dat een klein gebied in het achterste deel van de linker temporale kwab minder actief was in sociaal angstige mensen bij het bekijken van scenes waarin de proefpersonen zich sterk zouden inleven in de personages. Dit gebied is vaak betrokken bij het interpreteren van binnenkomende sociale cues in het licht van sociale kennis die je al bezit, wat dus verzwakt zou kunnen zijn in sociaal angstige mensen. Als we kijken naar hoe wisselend de hersenactiviteit tussen proefpersonen is binnen dezelfde groep, zien we dat bepaalde stukken van de film en hersengebieden zowel meer als minder wisselend zijn in sociaal angstige mensen in vergelijking met niet sociaal angstige mensen. Dit wijst erop dat er een mogelijke automatische bias is in het verwerken van sociale cues in sociaal angstige mensen die hun andere manier van denken kan verklaren. 

% chapter 4
Vanwege de veelvoorkomende vertragingen in taalontwikkeling in autisme hebben we in hoofdstuk~\ref{ch:language_asc} onderzocht hoe de hersensignalen eruit zien als autistische en neurotypische mensen zinnen proberen te begrijpen. Er is namelijk nog niet eerder onderzoek gedaan naar de vraag of zinsopbouw anders verwerkt wordt in de hersenen en vooral of bepaalde signalen eerder of later plaatsvinden. In dit experiment lazen de twee groepen proefpersonen zinnen met of zonder zinstructuur terwijl hun hersengolven werd gemeten met EEG (Electroencephalography). De resultaten bevestigden eerder onderzoek dat hersengolven in de beta band (14 - 20 Hz) een goede signatuur bleken van het verwerken van zinsopbouw, maar deze en andere hersengolven lieten dezelfde patronen zien in autistische en neurotypische proefpersonen. Op basis van eerder onderzoek met een andere methode waren we ook ge\"interesseerd of de hersengolven even sterk aanwezig waren in beide hersenhelften. Dit was het geval, en bovendien niet verschillend tussen de twee proefpersoongroepen of over de tijd heen die het kostte om een zin te lezen.

% conclusion
De bevindingen in dit proefschrift laten zien dat eerder beschreven verschillen tussen autistische en neurotypische personen niet universeel aanwezig zijn in alle autistische personen. We weten nu dus dat het onjuist is om te praten over een algemene tekortkoming in het mentalizingvermogen van autistische personen. Ook zijn veelgenoemde neurale kenmerken tijdens taalbegrip in autisme niet altijd aangetast. Verder hebben we ontdekt dat het nuttig is om mentale vaardigheden opnieuw te testen in een setting die meer overeenkomen met hoe deze in de echte wereld toegepast worden. Hiermee hebben we in autisme en sociale angst aangetood dat mentalizingvermogen en het interpreteren van social interacties anders kan zijn dan aanvankelijk bekend was. Meer kennis over de specifieke aspecten die sociale informatie interpreteren anders maakt in deze condities is noodzakelijk voor een beter begrip hierover, waarvoor het onderzoek in dit proefschrift een weg heeft kunnen banen.
