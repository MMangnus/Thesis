\section{Availability and Future Directions}
We have presented a new tool to visually construct an analysis pipeline. Subsequently, Porcupine automatically generates the analysis code, and provides a way of running and sharing such analyses. We see this as an important tool and a stepping stone on the path to doing more reproducible and open science. Additionally, this gives researchers a better oversight of their analysis pipeline, allowing for greater ease of developing, understanding, and communicating complex analyses.

Porcupine provides two independent functionalities that dovetail to allow users to more easily take part in reproducible neuroimaging research. They are (1) a graphical user interface for the visual design of analysis pipelines and (2) a framework for the automated creation of docker images to execute and share the designed analysis. 

We anticipate that the ability to design processing pipelines visually instead of programmatically \change{cuts}{will cut} the novice user's learning phase by a considerable amount of time by facilitating understanding and development. The ease of use of a Graphical User Interface (GUI) implementation extends and complements Nipype's flexibility. Thus, it invites researchers to mix and match different tools, and adhere less stringently to the exclusive use of the tools of any given toolbox ecosystem. This flexibility enhances the possible sophistication of processing pipelines, and could for instance be helpful in cross-modal research or multi-site research. Additionally, it may nudge method developers to write new tools in a way that easily integrates with the Nipype and Porcupine structure.

The emphasis that Porcupine puts on visual development of analyses makes it easier to communicate a methods section visually rather than in writing. We foresee that researchers may prefer explicity sharing the created .pork files and the Nipype pipelines that are created from them, instead of solely relying on written descriptions of their methods. Yet another use case for Porcupine is the easy definition of proposed processing workflows for preregistered studies.

Importantly, Porcupine attempts to reduce the steepness of the learning curve that is inherent to the use of complex analysis, by providing a more structured and systematic approach to pipeline creation. It separates the skill of building a conceptual analysis pipeline from the skill of coding this in the appropriate programming language. This places Porcupine in a position to aid in the education of novice neuroimaging researchers, as it allows them to focus on the logic of their processing instead of the creation of the code for the processing - greatly improving and accelerating their understanding of the different steps involved in the preprocessing of neuroimaging data. At the same time, it allows more experienced researchers to spend more time on \change[Reviewer 2]{theconceptual}{the conceptual} side than on implementational side.

Having allowed for the visual design of a pipeline for the preprocessing or analysis of a neuroimaging dataset, the reproducible execution of this pipeline is another step that Porcupine facilitates. By flexibly creating a Docker image tailored to the different preprocessing steps defined visually in the GUI, Porcupine allows the user to share not only the definition of the pipeline but also its execution environment. This step removes the overhead of having to manually install the desired operating system with the matching distribution of MRI analysis software. This final step greatly facilitates the reproducibility of reported results, and is part of a general evolution of the field towards easily shareable and repeatable analyses. 

The generated Docker image can be made High Performance Computing aware \remove{with singularity} by means of dedicated tools such as \href{https://github.com/singularityware/docker2singularity}{docker2singularity}. Alternatively, with only trivial additions to the Dockerfile, it can be transformed into a BIDS app \cite{Gorgolewski2017}. A detailed explanation for doing this can be found on our \href{https://timvanmourik.github.io/Porcupine/documentation/advanced/make-a-bids-app}{website}. An automatic and direct way of creating \add{this} has not yet been implemented. Additionally, integrating support for standardised workflow file formats, such as the Common Workflow Language \cite{Amstutz2016} could further add to Porcupine's aim of reproducibility. Another point of improvement is a functionality to embed pipelines within pipelines. Currently, a complicated pipeline\remove[Reviewer 2]{s} does full justice to the term `spaghetti code', and the number of nodes and links may easily compromise the visual aid in understanding; the very purpose for which Porcupine was created. This may easily be solved by compartmentalising pipelines into logical units by providing an embedded structure.

We intend Porcupine to be a strong aid for doing better, more reproducible and shareable science. By bridging the gap between a conceptual and implementational level of the analysis, we give scientists a better oversight of their pipeline and aid them in developing and communicating their work. We provide extensive and intuitive documentation and a wide range of examples to give users a frictionless start to use Porcupine. We look forward to adding more functionality and\change{support for}{supporting} more toolboxes in the near future.
