\chapter{Preserved Spontaneous Mentalizing amid Reduced Intersubject Variability in Autism during a Movie Narrative}
\label{ch:mentalizing_asc}

\vspace{-1cm}
\begin{center}
    \large\textit{Abstract}
\end{center} 

{\abstractfont 
While individuals with autism often face challenges in everyday social interactions, they may demonstrate proficiency in structured Theory of Mind (ToM) tasks that assess their ability to infer others' mental states. Using functional MRI and pupillometry, we investigated whether these discrepancies stem from diminished spontaneous mentalizing or broader difficulties in unstructured contexts. Fifty-two adults diagnosed with autism and 52 neurotypical controls viewed 'Partly Cloudy', a nonverbal animated film with a dynamic social narrative known to engage the ToM brain network during specific scenes. Analysis focused on comparing brain and pupil responses to these ToM events. Additionally, dynamic intersubject correlations explored the variability of these responses throughout the film. Both groups showed similar brain and pupil responses to ToM events and provided comparable descriptions of the characters' mental states. However, participants with autism exhibited significantly stronger correlations in their responses across the film's social narrative, indicating reduced inter-individual variability. This distinct pattern emerged well before any ToM events and involved brain regions beyond the ToM network. Our findings provide functional evidence of spontaneous mentalizing in autism, demonstrating this capacity in a context affording but not requiring mentalizing. Rather than responses to ToM events, a novel neurocognitive signature - inter-individual variability in brain and pupil responses to evolving social narratives - differentiated neurotypical individuals from those with autism. These results suggest that idiosyncratic narrative processing in unstructured settings, a common element of everyday social interactions, may offer a more sensitive scenario for understanding the autistic mind. 
} 

\vspace{2cm}
This chapter is adapted from \bibentry{mangnus2024BPCNNI}

\thispagestyle{empty}

\newpage

\section*{Introduction}
Autism Spectrum Condition (ASC) is a neurodevelopmental disorder marked by difficulties in social communication and interaction across multiple contexts \citep{apa2013}. These difficulties are frequently associated with Theory of Mind (ToM) - the ability to attribute mental states to oneself and others, also known as mentalizing \citep{premack1978,wimmer1983}. While initial findings suggested diminished ToM abilities in autism \citep{baron-cohen1985,happe1994}, subsequent studies have painted a more complex picture, owing to a variety of factors. 

First, the reliability of established ToM assessments, such as the False Belief Test, Strange Stories, and the Reading the Mind in the Eyes Test, has increasingly come under scrutiny. These tests have shown inconsistent effect sizes when compared to earlier, smaller-scale studies and exhibit limited correlations with one another, despite their aim to measure similar or identical ToM constructs \citep{gernsbacher2019,higgins2024,schaafsma2015,yeung2024}. Second, many studies do not adequately match autistic individuals with neurotypical controls based on language abilities, which are crucial to varying degrees for these tasks \citep{betz2019}. This lack in language matching is particularly significant given the generally lower performance in verbal learning and memory observed within the autism population \citep{velikonja2019}. Lastly, recent research highlights the remarkable proficiency of autistic individuals in ToM tasks, especially in situations requiring strategic mental state reasoning \citep{bowler1992,pantelis2017}, such as deception \citep{vantiel2021}.

Faced with these empirical challenges, researchers have turned to more implicit ToM measures. This includes the analysis of anticipatory eye movements concerning an actor's false beliefs \citep{senju2009}, and measuring neural activity in the Animated Triangles task, where participants attribute mental states to moving shapes \citep{abell2000}. Recent findings indicate that while autistic individuals do show anticipatory gaze responses, these responses tend to be generally slower, regardless of whether the character's beliefs were true or false \citep{glenwright2021,schuwerk2016}. In the Animated Triangles task, both autistic and neurotypical participants show comparable brain activation in key ToM regions, such as the medial prefrontal cortex (mPFC), temporoparietal junction (TPJ), and precuneus \citep{moessnang2020}. However, individuals with autism typically underperform in both the mentalizing and non-mentalizing conditions of this task, which, combined with their generally slower anticipatory gaze responses, points to broader difficulties in implicit task settings \citep{wilson2021}. The extent to which autistic individuals engage in spontaneous mentalizing, particularly in unstructured settings lacking an explicitly defined task, as is common in everyday social interactions, remains largely unknown.

In this study combining fMRI and pupillometry, we aimed to bridge this knowledge gap by investigating how individuals with and without autism respond to mental state events embedded in a dynamic movie narrative. Additionally, we sought to provide insights into how these individuals process stimuli in less structured environments that more closely resemble everyday social interactions. Everyday interactions require a continuous assessment of stimuli within an evolving narrative \citep{goffman1974,johnson2023,stolk2022}, as seen in how even seemingly minor behaviors like a voluntary cough or a brief silence can carry significant implications in certain contexts \citep{kendon1994}. Failure to recognize these narrative cues could lead to misunderstandings in scenarios that require high levels of interpretation, such as irony or sarcasm, areas known to pose challenges for autistic individuals \citep{deliens2018,zalla2014}. However, capturing this narrative processing is challenging, as it likely unfolds differently over time among individuals exposed to the same stimuli. We reasoned that narrative processing could manifest as inter-individual variability in responses to movie stimuli, particularly diminished among those less inclined to interpret stimuli through a narrative lens \citep{chang2021,finn2018,owen2023,zhang2022}. We investigated this possibility using intersubject correlation analysis \citep{hasson2004}, applying it to two complementary methods for assessing cognitive processing: brain imaging and measurements of pupil size \citep{beatty1982}.

To probe spontaneous mentalizing and narrative processing, we recorded participants' brain and pupil responses as they viewed the nonverbal animated movie 'Partly Cloudy'\citep{jacoby2016,paunov2019}. Although movies cannot fully replicate the complexities of real-world social interactions \citep{wheatley2019}, they offer an effective platform for immersing participants in an evolving narrative under uniform stimulus conditions. To assess differences in narrative processing between autistic and neurotypical participants at various points during the movie, we augmented our intersubject correlation analysis with a dynamic sliding window technique and an adaptive clustering algorithm \citep{maris2007}. The movie Partly Cloudy was chosen for its proven ability to evoke neural activations within the ToM network through distinct mental state events \citep{jacoby2016,richardson2018}, making it suitable for evaluating spontaneous mentalizing. After viewing, we analyzed participants' verbal descriptions of the movie, focusing on their use of language related to mental states. This experimental approach enabled us to simultaneously probe spontaneous mentalizing and narrative processing, providing insights into how these cognitive functions interact in individuals with autism.

\section*{Methods and Materials}
\subsection*{Preregistration and Data availability}
The study comprised two sets of preregistered analyses \citep{mangnus2022}. The first set focused on spontaneous pupil and brain responses to movie events anticipated to elicit mental state inferences, complemented by a questionnaire that examined participants' use of related vocabulary in describing the movie. The second set of analyses explored dynamic intersubject correlations to investigate idiosyncratic narrative processing throughout the entire movie. Unlike the event-related analyses, these exploratory analyses were not limited to specific movie segments. All resulting pupillometry and fMRI data are publicly available for further research \citep{mangnus2024dataset}.

\subsection*{Participants}
The study enrolled 104 participants, divided equally into 52 adults diagnosed with Autism Spectrum Disorder (ASC) and 52 neurotypical controls (NT). Recruitment occurred through Radboud University's database, social media, campus postings, and outpatient clinics in Nijmegen and Arnhem, the Netherlands. Eligibility for the ASC group required a formal diagnosis from a clinician \citep{apa2013}, while exclusion criteria for all included the use of psychotropic drugs, severe cognitive impairment, systemic diseases, or neurological treatment history. As shown in Table~\ref{tab:ppt_stats}, both groups were demographically matched for gender, age (Kullback-Leibler divergence = .05, \textit{F}-test = .29), and both verbal and nonverbal IQ, verified through the Similarities and Vocabulary subscales of the Wechsler Adult Intelligence Scale (WAIS-III \citep{wechsler1997}; KL = .02, \textit{F}-test = .74) and Raven's Progressive Matrices (RPM \citep{raven1989}; KL = .01, \textit{F}-test = .63). ASC participants notably scored higher on the Autism-Spectrum Quotient (AQ-50 \citep{baron-cohen2001AQ}), confirming group distinctions. Data collection involved MRI scans (\textit{n} = 104) and pupillometry (\textit{n} = 100) while participants viewed the film, followed by a post-viewing questionnaire completed by most (\textit{n} = 101). All participants provided written informed consent, approved by the local ethics committee (CMO region Arnhem-Nijmegen, file number 2019-6059), and received compensation for participation.

\begin{table}[ht]
    \captionsetup{justification=raggedright, singlelinecheck=false, font = normal} % Left-align the caption
    \caption{Demographic data}
    \label{tab:ppt_stats}
    \setlength{\tabcolsep}{12pt}
    \renewcommand{\arraystretch}{1.5} % Adjust row spacing (default is 1.0)
    \begin{tabular}{llll}
    \hline
    \textbf{} & \textit{ASC Group} & \textit{NT Group} & \textit{Group Difference} \\
    \hline
    N (male:female) & 52 (23:29) & 52 (20:32) & \textit{$\chi$\textsuperscript{2}}(1, \textit{N} = 104) = 1.73, \textit{p} = .19 \\
    Age (years) & 27.7 (6.3) & 25.9 (5.5) & \textit{t}(102) = 1.74, \textit{p} = .08 \\
    Verbal IQ (WAIS-III) & 126 (16) & 124 (16) & \textit{t}(102) = 0.51, \textit{p} = .61 \\
    Nonverbal IQ (RPM) & 103 (9) & 104 (11) & \textit{t}(102) = -0.46, \textit{p} = .64 \\
    Autism Quotient & 31 (9) & 12 (6) & \textit{t}(102) = 12.2, \textit{p} < .001 \\
    \hline
    \multicolumn{4}{l}{\small{Values are given as Mean (Standard Deviation). ASC, Autism Spectrum Condition; NT, Neurotypical.}} \\
    \end{tabular}
\end{table}
    

\subsection*{Experimental design }
Participants watched the 5-minute and 45-second animated short Partly Cloudy, depicting the evolving friendship between a stork and a cloud through various social interactions integral to its narrative. After viewing, they described the plot to assess narrative comprehension and articulation of characters' mental states. Descriptions were analyzed by independent raters, categorizing words into mental state terms or other content-related categories, using the mental state word list from Bang et al. \citep{bang2013}. An independent t-test compared the frequency of mental state words used between groups. The movie featured three types of events, coded by the original research team that introduced it as a Theory of Mind localizer task \citep{jacoby2016}. These included \textit{Mental}, \textit{Pain}, and \textit{Control} events, each annotated on the movie's timeline in Fig~\ref{fig:task-fig}. \textit{Mental} events, totaling 44 seconds across 4 events, were expected to elicit inferences about characters' mental states, depicting scenarios like characters feeling distressed while observing others enjoy a cheerful interaction or mistakenly perceiving betrayal by a friend. \textit{Pain} events, totaling 26 seconds across 7 events, depicted instances where characters experienced physical discomfort, such as being shocked by an eel or bitten by a crocodile. \textit{Control} events, totaling 24 seconds across 3 events, showcased serene moments like birds in flights or panoramic views of clouds. As detailed in later sections, these categorizations facilitated the analysis of ToM-related pupillary and neural responses.

\begin{figure}[!ht]
	\centering
    \makebox[\textwidth][c]{\includegraphics[width=1.05\textwidth]{./Chapters/02_MentalizingASC/Images/TaskFig.eps}}
	\caption{Autistic and neurotypical participants viewed a six-minute animated movie portraying the evolving friendship between a stork and a cloud, while their pupil and brain responses were recorded. Originally used by Jacoby et al. (2016) as a Theory of Mind localizer, the movie includes three types of events: Mental, Pain and Control, each marked on the movie timeline. Mental and Pain events were expected to prompt inferences about characters' mental and physical states, whereas Control events featured no characters in the foreground. The event images shown were generated using Copilot in Bing for copyright purposes to closely resemble scenes from the movie.}
    \vspace*{-10pt}
	\label{fig:task-fig}
\end{figure}

%[width=.9\textwidth]



\subsection*{Pupillometry and MRI data acquisition}
Pupil size was continuously tracked using an Eyelink 1000 plus eyetracker at 1000 Hz. MRI data were acquired using a Siemens 3T MRI-scanner with a 32-channel head coil. Structural images were obtained with a T1 MPRAGE sequence (TR = 2200 ms, TI = 1100 ms, TE = 2.6 ms, flip angle = 11\textdegree, 
voxel size = 0.8 mm\textsuperscript{3}, acceleration factor = 2). Functional images were acquired with a multi-band multi-echo sequence (TR = 1500 ms, TE = 13.4/34.8/56.2 ms, flip angle = 75\textdegree, voxel size = 2.5 mm\textsuperscript{3}, acceleration factor = 2). Analysis of head movement through framewise displacement (FD) showed no significant differences between the ASC and NT groups. Mean FD was 0.15 \textpm{}{} 0.05 for ASC and 0.16 \textpm{}{} 0.09 for NT (M \textpm{} SD; \textit{t}(102) = -0.78, \textit{p} = .43), with maximum FD values at 0.81 \textpm{} 0.79 for ASC and 1.03 \textpm{} 1.31 for NT (\textit{t}(102) = -1.09, \textit{p} = .28). Additionally, assessments of total head motion, calculated from translation and rotation during the realignment process, indicated no significant differences (Translation: 109.5 \textpm{} 74.6 vs. 114.3 \textpm{} 99.5, \textit{t}(102) = 0.28, \textit{p} = .78; Rotation: 2.0 \textpm{} 0.98 vs. 2.2 \textpm{} 1.2 , \textit{t}(102) = 0.74, \textit{p} = .46).

\subsection*{Pupillometry data analysis}
Pupillometry data underwent preprocessing with a combination of established and custom MATLAB routines. Blinks were removed using a noise-based detection algorithm \citep{hershman2018}. Squints, marked by unusually small pupil sizes \citep{mathot2018}, and gaze jumps, indicative of excessive translational eye movements, were both identified and eliminated through visual inspection. After these adjustments, 89.3\% \textpm{} 9.4\% of the data remained usable for the ASC group and 88.4\% \textpm{} 10.6\% for the NT group (\textit{t}(98) = 0.48, \textit{p} = .63). Fixations and saccades were distinguished using an adaptive velocity threshold \citep{nystrom2010}. Pupil timeseries were normalized (\textit{z}-scored) and adjusted for global luminance fluctuations modeled using the \textit{lm()} function from the R \textit{stats} package \citep{bates2015} up to the 5th polynomial order, validated with data from five randomly selected participants. Luminance was quantified using RGB values based on the Rec. 709 formula \citep{itu2002}. 

Event-related pupil responses were analyzed through a 3x2 mixed-design ANOVA with mean pupil size as the dependent variable, and event conditions (\textit{Mental, Pain, Control}) and participant group status (ASC, NT) as factors. Tukey's Honest Significant Difference tests further investigated ToM-related contrasts, specifically comparing \textit{Mental} to both \textit{Pain} and \textit{Control} conditions. To verify the robustness of our findings against variations in event timing, a control analysis was conducted by incorporating an additional time-based regressor into the \textit{lm()} function during preprocessing. This adjustment, which incremented by one every second, did not influence the main findings.

Intersubject variability was assessed using dynamic intersubject correlation analysis of the pupil timeseries. Employing a leave-one-out approach, each participant's timeseries was correlated with the composite average timeseries of all other group members \citep{nastase2019}. This analysis was conducted using a 30-second sliding window at 100 ms intervals, generating a correlation timeseries for the entire movie duration per participant. All correlation timeseries were Fisher \textit{z}-transformed and subjected to a nonparametric cluster-based permutation test \citep{maris2007}. This test addresses the multiple comparisons problem in timeseries analysis by clustering significant neighboring data points. These clusters are tested against a null distribution formed by randomly shuffling participant labels and recalculating statistics, allowing the identification of specific timepoints where significant differences in pupil response variability between groups occurred, while effectively controlling for false positives. Statistical testing was performed using a two-sided independent samples \textit{t}-test with 10,000 permutations to establish the null distribution. Clusters that reached a Monte-Carlo \textit{p}-value of .05 or less were considered statistically significant.

\subsection*{fMRI data analysis}
Functional images were preprocessed using SPM12, initially consolidating multiple echoes into single volumes through echo-weighted combinations. These volumes were realigned to the initial image using rigid-body transformations and 2nd degree B-spline interpolation, and subsequently unwarped with participant-specific fieldmaps to minimize spatial distortions and signal dropout. Anatomical images were coregistered to the mean functional image and segmented into gray matter, white matter, and cerebral spinal fluid categories using SPM's tissue probability maps, enabling normalization to MNI space. An 8 mm full-width at half-maximum kernel was applied or spatial smoothing. First-level regressors estimated activations for \textit{Mental, Pain,} and \textit{Control} events, and included adjustments for head movement (using squared and cubic terms, along with first and second derivatives) and tissue signal intensities. A 0.6 threshold masking was applied to ensure optimal brain coverage.

Two 2-by-2 mixed-design ANOVAs were conducted to analyze Theory of Mind (ToM) activations across \textit{Mental, Pain,} and \textit{Control} conditions and participant groups (ASC, NT). The first contrast (\textit{Mental} > \textit{Pain}) sought to confirm previous findings of enhanced ToM network activity \citep{jacoby2016}, while the second (\textit{Mental} > \textit{Control}) extended these findings to scenes lacking foreground characters. Results were subjected to whole-brain, cluster-level correction (\textit{p\textsubscript{FWE}} < .05), with anatomical locations identified using the SPM Anatomy Toolbox \citep{eickhoff2005}. Additionally, a region of interest (ROI) analysis focused on three peak brain regions, complemented by Bayesian analysis \citep{jasp2022} to evaluate consistent neural activation within the ToM network across groups, reporting evidence with Bayes Factors (BFs). Robustness against event timing variations was confirmed through a control analysis adding a first-level regressor that incremented with each volume.

Intersubject variability at the neural level was assessed using dynamic intersubject correlation analysis of voxel timeseries, adjusted for head movement and tissue signals. Mirroring the pupillometry analysis, a leave-one-out and sliding window method generated whole-brain correlation timeseries for each participant. For computational efficiency, the data were spatially and temporally downsampled by a factor of three, resulting in a voxel resolution of 7.5 mm and a sampling interval of 4.5 seconds. These data underwent a nonparametric cluster-based permutation test \citep{maris2007}, which corrected for multiple comparisons across voxels and timepoints using a two-sided independent samples \textit{t}-test with 10,000 permutations (Monte-Carlo \textit{p} < .05). The analysis was further refined with a spatially adaptive clustering algorithm, identifying variations in brain response patterns between groups. The spatial distribution of the identified spatiotemporal brain cluster was visualized by summing t-values across all timepoints, producing a three-dimensional representation of significant variability. The degree of overlap between this cluster and the ToM network was quantified by comparing their spatial volumes. Lastly, a supplementary analysis without the sliding window approach calculated correlations across the entire voxel timeseries, pinpointing a smaller cluster in the left supramarginal gyrus. In line with the main findings, this cluster demonstrated reduced intersubject variability in the ASC group.

\section*{Results}
\subsection*{Post-viewing movie descriptions }
After the film viewing, participants were asked to recount the story of the stork and cloud's evolving friendship in their own words. We analyzed these descriptions for the use of mental state words and other content-related terms (Fig~\ref{fig:beh-pupil}a). Statistical analysis revealed no significant differences in the frequency of mental state word usage between the autistic and neurotypical participants (M\textsubscript{ASC} = 0.063, M\textsubscript{NT} = 0.054, \textit{t}(99) = 0.84, \textit{p} = .40; Fig. 2b), with a Bayes Factor in favor of the null hypothesis (BF\textsubscript{Null} = 3.49). This finding suggests that both autistic and neurotypical groups engaged similarly with the mental states depicted in the film.

\begin{figure}[!ht]
    \vspace*{10pt}
	\centering
    \makebox[\textwidth][c]{\includegraphics[width=1.05\textwidth]{./Chapters/02_MentalizingASC/Images/BehPupil.eps}}
	\caption{Comparable mental state descriptions and pupil responses across groups. (a) Word Cloud depicting the frequency of mental state-related vocabulary used by participants in their post-viewing movie descriptions. The size of each word corresponds to its frequency of use. (b) Bar graphs displaying the ratio of mental state words to other content-related terms in these descriptions, showing no significant differences between the two groups. (c) Pupil responses across all event conditions, revealing similar response patterns in both autistic and neurotypical groups. Asterisks denote significant differences between event conditions (all \textit{p} < .001), with no significant differences observed between groups. Outliers are represented by dots, while whiskers display a 1.5 inter-quartile range.}
    \vspace*{-10pt}
	\label{fig:beh-pupil-asc}
\end{figure}



\subsection*{Event-related pupil responses}
Participants' pupil sizes were continuously tracked throughout the movie to assess responses to events expected to elicit mental state inferences, such as the cloud reflecting on the stork's actions. We also analyzed responses to events evoking physical state inferences, like the stork experiencing pain, as well as to control scenes devoid of character interactions. Pupillometry analysis identified distinct responses to these three event types (\textit{F}(2,196) = 73.0, \textit{p} < .001, BF\textsubscript{Null} = 0.00; Fig~\ref{fig:beh-pupil}c), with the largest pupil dilation occurring during \textit{Pain} events (M = 0.22, \textit{z}-score), followed by \textit{Mental} events (M = -0.12) and \textit{Control} events (M = -0.20). Comparisons of pupil responses between autistic and neurotypical participants revealed no significant differences (\textit{F}(1,98) = 0.03, \textit{p} = .86, BF\textsubscript{Null} = 7.8), nor were there significant interaction effects between participant groups and event types (\textit{F}(2,196) = 2.1, \textit{p} = .12, BF\textsubscript{Null} = 2.5). This suggests that both groups reacted similarly to the different event types in the movie.

\subsection*{Event-related brain responses}
A whole-brain fMRI analysis was conducted to examine neural activations during \textit{Mental} events in comparison to \textit{Control} and \textit{Pain} events. As depicted in Fig~\ref{fig:fmri-results} a, this analysis identified robust activation in keys areas of the Theory of Mind (ToM) network \citep{schurz2014}, specifically in the right and left temporoparietal junction (rTPJ: xyz\textsubscript{MNI} = [48, -62, 32], \textit{t} = 16.85, \textit{p\textsubscript{FWE}} < 0.001; lTPJ: [-46, -62, 32], \textit{t} = 16.92, \textit{p\textsubscript{FWE}} < 0.001), the precuneus ([6, -64, 40], \textit{t} = 20.64, \textit{p\textsubscript{FWE}} < 0.001), medial prefrontal cortex (mPFC: [-6, 52, 38], \textit{t} = 7.33, \textit{p\textsubscript{FWE}} < 0.001), and left middle temporal gyrus ([-52, 2, -26], \textit{t} = 11.51, \textit{p\textsubscript{FWE}} < 0.001). Echoing the pupillometry findings, no significant differences in neural activation were observed between autistic and neurotypical participants. Complementary region of interest (ROI) analysis (Fig~\ref{fig:fmri-results}b) and Bayesian analysis reinforced these findings, providing evidence favoring the null hypothesis over alternative models suggesting group differences. This was consistently demonstrated across all evaluated ROIs for both \textit{Mental} > \textit{Pain} contrasts (rTPJ: BF\textsubscript{Null} = 4.83, precuneus: BF\textsubscript{Null} = 1.73, mPFC: BF\textsubscript{Null} = 4.01) and \textit{Mental} > \textit{Control} contrasts (rTPJ: BF\textsubscript{Null} = 3.43, precuneus: BF\textsubscript{Null} = 3.63,  mPFC: BF\textsubscript{Null} = 4.71), underscoring a similarity in neural processing of mental states among autistic and neurotypical individuals.

\begin{figure}[!ht]
	\centering
    \includegraphics[width=1\textwidth,trim={0 9cm 0 0},clip=true]{./Chapters/02_MentalizingASC/Images/FMRIResults.eps}
	\caption{Comparable neural activation patterns during mental state events across groups. (a) Brain sections showing whole-brain responses specific to mental state events, as assessed through the Mental > Control and Mental > Pain contrasts. There were no significant differences in activation for either contrast between the autistic and neurotypical groups. (b) Box plots depicting the contrast estimates for the Mental > Control (in blue) and Mental > Pain (in red) comparisons across various Regions of Interest (ROI) for both groups. rTPJ and mPFC were selected as ROIs based on mentalizing literature, while the precuneus was selected because of the peak voxel being located in that region for both mentalizing contrasts. Across all ROIs and contrasts, no significant differences were observed between the groups. Outliers are represented by dots, while whiskers display a 1.5 inter-quartile range.}
    \vspace*{-10pt}
	\label{fig:fmri-results}
\end{figure}

%\makebox[\textwidth][c]{



\subsection*{Movie-driven variability in pupil responses }
Having observed comparable pupil and brain responses to mental state events and similar verbal descriptions from both autistic and neurotypical participants, we expanded our investigation to narrative processing differences across the entire movie through dynamic intersubject correlation analysis of the pupil timeseries. This analysis revealed an interval from 40 to 71 seconds where individuals with autism demonstrated significantly stronger correlations in pupil responses, indicating reduced intersubject variability compared to neurotypical participants (M\textsubscript{ASC} = 0.61, M\textsubscript{NT} = 0.55, \textit{cluster stat} = 777, \textit{p} = .045; Fig~\ref{fig:isc-pupil}). This interval preceded any mental state events and coincided with scenes featuring storks flying through the air and clouds morphing into baby animals. Although this reduced variability continued throughout the film, it did not reach statistical significance outside this interval after adjusting for multiple comparisons. Further, differences in variability were not due to variations in saccadic eye movements, as their frequency and variability remained consistent between groups (Fig~\ref{fig:saccades-suppl}).

\begin{figure}[!ht]
    \vspace*{5pt}
	\centering
    \includegraphics[width=1\textwidth,clip=true]{./Chapters/02_MentalizingASC/Images/ISCPupil.eps}
	\caption{Reduced pupil response variability in autistic participants. Dynamic intersubject correlation analysis initially showed comparably high levels of correlation in both autistic and neurotypical groups at the start of the movie. However, a significant divergence emerged around the 40-second mark, where autistic individuals showed stronger correlations in their pupil responses, indicating reduced intersubject variability, compared to neurotypical participants. This pattern of reduced variability emerged well before the mental state events highlighted in red. Solid lines delineate statistically significant intervals, as determined by a cluster-based permutation test.}
    \vspace*{-15pt}
	\label{fig:isc-pupil-asc}
\end{figure}





\subsection*{Movie-driven variability in brain responses}
When applied to the fMRI data, dynamic intersubject correlation analysis identified a spatiotemporal cluster where individuals with autism showed significantly stronger correlations in their brain responses compared to neurotypical participants (Fig~\ref{fig:isc-fmri-time}). This cluster spanned the entire movie (\textit{cluster stat} = 1332, \textit{p} = .002) and included peaks in the right and left supramarginal gyrus (rSMG: xyz\textsubscript{MNI} = [52, -34, 32], \textit{t\textsubscript{max}} = 3.88; lSMG: [-54, -40, 32], \textit{t\textsubscript{max}} = 4.54), the right inferior temporal gyrus (rITG: [54, -22, -28], \textit{t\textsubscript{max}} = 5.90), and the left calcarine gyrus (lCG: [6, -102, -10], \textit{t\textsubscript{max}} = 4.20). The reduced variability was consistent across these regions for most of the movie (Fig~\ref{fig:isc-fmri-peak-suppl}), and it emerged well before and continued after any mental state events. The cluster's overlap with the ToM network was minimal, comprising less than 20\% of the total cluster size (Fig~\ref{fig:isc-fmri-overlap}). This indicates that the observed reduced variability in brain responses among autistic participants extends beyond regions involved in mental state processing, suggesting broader differences in neural processing between the groups.

\begin{figure}[!ht]
	\centering
    \makebox[\textwidth][c]{\includegraphics[width=1.05\textwidth,clip=true]{./Chapters/02_MentalizingASC/Images/ISCFMRITime.eps}}
	\caption{Reduced brain response variability in autistic participants. Dynamic intersubject correlation analysis identified a spatiotemporal brain cluster where autistic individuals demonstrated significantly stronger correlations in their brain responses, indicating reduced intersubject variability, compared to neurotypical participants. This cluster persisted throughout the movie and featured peaks in the right and left supramarginal gyrus, the right inferior temporal gyrus, and the left calcarine gyrus. Echoing the pupillometry data, this pattern of reduced variability in autistic participants emerged well before the mental state events highlighted in red. LH, left hemisphere; RH, right hemisphere.}
    \vspace*{-10pt}
	\label{fig:isc-fmri-time-asc}
\end{figure}





\begin{figure}[!ht]
	\centering
    \includegraphics[width=1\textwidth,clip=true]{./Chapters/02_MentalizingASC/Images/ISCFMRIOverlap.eps}
	\caption{Limited spatial overlap between brain regions showing reduced intersubject variability in autistic participants and mentalizing activation. The overlap comprised less than 20\% of the total cluster size. For clarity, the cluster is visualized using a cumulative \textit{t}-value threshold of 20 or higher. Lateral brain images provide an overlay with a search distance of 10 mm. }
    \vspace*{-10pt}
	\label{fig:isc-fmri-overlap}
\end{figure}





\section*{Discussion}
Using fMRI and pupillometry, this study provides functional evidence of spontaneous mentalizing abilities in individuals with autism. Compared to neurotypical controls matched for gender, age, and both verbal and nonverbal IQ, individuals with autism exhibited similar brain and pupil responses during movie scenes known to activate the Theory of Mind network \citep{jacoby2016,richardson2018}. Activity in the ToM network was enhanced during scenes that encouraged viewers to contemplate the actions and mental states of depicted characters. Conversely, scenes prompting physical state inferences, such as characters experiencing physical discomfort, or control scenes lacking central characters, resulted in weaker ToM activations. Verbal descriptions provided by participants after the movie corroborated these findings, indicating an engagement with the mental state events depicted in the movie on par with that of neurotypical controls. These results extend prior evidence of preserved mentalizing in autism \citep{moessnang2020,dufour2013}, showcasing this capacity in a situation affording but not requiring mentalizing.

While individuals with and without ASC showed comparable brain and pupil responses during mental state events, dynamic intersubject correlation analysis revealed significant differences in the correlation of these responses over extended movie intervals. Participants with autism exhibited significantly stronger correlations, indicating reduced inter-individual variability, across several brain regions outside the ToM network. These regions included the right and left supramarginal gyrus, linked to empathic judgment \citep{silani2013,wada2021}, the right inferior temporal gyrus, associated with narrative comprehension \citep{youssofzadeh2022}, and the left calcarine gyrus, crucial for visual processing \citep{woldorff2002}. This reduced variability was not due to differences in bottom-up processing, as both autistic and neurotypical groups exhibited similar saccadic eye movement patterns throughout the film. The most pronounced differences emerged during early scenes featuring storks flying through the air and clouds morphing into baby animals, which likely introduced significant narrative ambiguity. This ambiguity may have prompted neurotypical viewers to idiosyncratically interpret how these visual elements fit into the evolving storyline, resulting in less consistent responses. In contrast, autistic participants' responses appeared to be more consistently aligned with the movie's stimuli, possibly reflecting a heightened focus on specific details rather than the broader narrative context \citep{losh2003,tager-flusberg1995,barnes2012,geelhand2020,koldewyn2014}. Importantly, these differences manifested in cognitive and neural processing rather than in eye movements, suggesting that the variability observed in neurotypical responses may represent a neurocognitive signature of top-down processing.

The neuroanatomical bases of the observed changes in response variability are in line with existing research on autism and social interaction. Prior studies have documented structural alterations in the gray matter of both the right and left supramarginal gyrus in individuals with autism \citep{brieber2007,ke2008,libero2014}, as well as reduced anatomical connectivity in the right inferior temporal cortex \citep{boets2018,koldewyn2014}. This region exhibits prolonged activations during tasks involving story comprehension and interactive communication \citep{youssofzadeh2022,stolk2013}, highlighting its role in integrating stimuli within an evolving narrative. This integration is crucial for complex social interactions, which necessitate the continuous assessment of diverse stimuli to maintain narrative coherence with others \citep{goffman1974,johnson2023,stolk2022}. A key direction for future research is to examine how response variability in the identified brain regions differs across various social contexts. Such investigations will not only help to illuminate the specific challenges faced by autistic individuals in everyday social situations, but could also inform the development of interventions by identifying environments that promote effective social interaction \citep{wadge2019}.

It is worth noting that our intersubject correlation patterns differ from previous studies demonstrating greater brain response variability in autistic individuals compared to neurotypicals \citep{byrge2015,hahamy2015,hasson2009,lyons2020,nunes2019,ou2022,pegado2020,salmi2013}. Several factors could explain these discrepancies. First, our study used an animated film with fictional characters, as opposed to the more realistic human portrayals in other studies. Although the type of characters may influence brain responses in autism \citep{atherton2018}, greater neural variability has been noted with fictional characters \citep{lyons2020}. Second, we implemented an adaptive clustering algorithm to identify spatiotemporal clusters of brain response variability within 30-second intervals. This approach is potentially more sensitive to brain response variations associated with subtle shifts in interpretation than whole-movie analyses, which emphasize consistent patterns over significantly longer durations (10 to 67 minutes). This approach may more effectively capture the hypothesized increased reliance on bottom-up sensory stimulation in autism \citep{pellicano2012}, possibly leading to less variability in neural signals related to narrative interpretation. More generally, our findings invite a reconsideration of theories that propose precise neural synchronization as a means to manage individual perspectives in daily interactions \citep{holroyd2022}. These theories suggest that aligning neural responses to external cues helps individuals achieve a common viewpoint, thereby facilitating social interaction \citep{hasson2012,mayo2021}. However, contrary to expectations based on their social difficulties, participants with autism in our study displayed stronger neural correlations when exposed to the same external stimuli, challenging the assumed role of precise neural synchronization in social interaction \citep{stolk2014}.

In conclusion, this study offers functional evidence of spontaneous mentalizing in autism, showcasing this capacity in a context affording but not requiring mental state inferences. More distinctively, our findings identify a novel neurocognitive signature - inter-individual variability in brain and pupil responses to evolving social narratives - that differentiates neurotypical individuals from those with autism. These results underscore the importance of idiosyncratic narrative processing in unstructured settings, a hallmark of everyday social interactions, as a potentially more sensitive framework for understanding the autistic mind.
\newpage
\section*{Supplementary information}

% table S1. 
\begin{table}[ht]
    \centering
    \captionsetup{justification=raggedright, singlelinecheck=false, font = normal} % Left-align the caption
    \setlength{\tabcolsep}{8pt} % Adjust column spacing if needed
    \renewcommand{\arraystretch}{1.5} % Adjust row spacing
    \caption{Results of the within-subject fMRI analyses related to the main effects of event type (Mental, Pain, Control).}
    \label{tab:fmri_anova}
    \begin{tabular}{lllccccc}
    \hline
    \textit{Contrast} & \textit{Anatomical location of peak voxel} & \textit{Cluster size} & \multicolumn{3}{c}{\textit{MNI coordinates}} & \textit{T-value} \\
     &  &  & \textit{x} & \textit{y} & \textit{z} & \textit{(df = 1, 102)} \\
    \hline
    Mental > Pain & Right precuneus & 19445 & 6 & -64 & 40 & 20.64 \\
     & Right superior frontal gyrus & 16376 & 28 & 26 & 54 & 12.76 \\
     & Left middle temporal gyrus & 2560 & -52 & 2 & -26 & 11.51 \\
     & Right middle temporal gyrus & 1782 & 60 & -12 & -20 & 10.14 \\
     & Right parahippocampal gyrus & 311 & 22 & -40 & -10 & 5.40 \\
    Mental > Control & Left precuneus & 31320 & -2 & -54 & 44 & 15.25 \\
     & Left middle temporal gyrus & 2525 & -56 & -8 & -20 & 10.58 \\
     & Left superior medial gyrus & 8939 & -6 & 52 & 38 & 7.33 \\
    \hline
    \end{tabular}
\end{table}
    
\newpage

\begin{figure}[!ht]
	\centering
    \includegraphics[width=1\textwidth,clip=true]{./Chapters/02_MentalizingASC/Images/SaccadesSuppl.eps}
	\caption{Frequency and variability of saccadic eye movements. (a) Mean number of saccades observed throughout the movie in both autistic and neurotypical individuals. No significant differences in the frequency of saccades were detected between the two groups. (b) Mean intersubject correlation coefficients representing the variability in the number of saccades across the movie among autistic and neurotypical individuals. No significant differences in saccade variability were observed between the two groups.}
    \vspace*{-10pt}
	\label{fig:saccades-suppl}
\end{figure}





\newpage

\begin{figure}[!ht]
	\centering
    \includegraphics[width=1\textwidth,clip=true]{./Chapters/02_MentalizingASC/Images/ISCFMRIPeaksSuppl.eps}
	\caption{Brain response variability across four key regions, including the right and left supramarginal gyrus (rSMG; lSMG), the right inferior temporal gyrus (rITG), and the left calcarine gyrus (lCG). \textit{T}-values were extracted from a 30 mm diameter sphere centered on the peak voxel. Variability in all four regions was statistically significant throughout most of the movie duration.}
    \vspace*{-10pt}
	\label{fig:isc-fmri-peak-suppl}
\end{figure}




