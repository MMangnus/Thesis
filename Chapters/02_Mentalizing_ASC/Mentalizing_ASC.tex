\chapter{Preserved Spontaneous Mentalizing amid Reduced Intersubject Variability in Autism during a Movie Narrative}
\label{ch:mentalizing_asc}

\section*{Abstract}
While individuals with autism often face challenges in everyday social interactions, they may demonstrate proficiency in structured Theory of Mind (ToM) tasks that assess their ability to infer others' mental states. Using functional MRI and pupillometry, we investigated whether these discrepancies stem from diminished spontaneous mentalizing or broader difficulties in unstructured contexts. Fifty-two adults diagnosed with autism and 52 neurotypical controls viewed 'Partly Cloudy', a nonverbal animated film with a dynamic social narrative known to engage the ToM brain network during specific scenes. Analysis focused on comparing brain and pupil responses to these ToM events. Additionally, dynamic intersubject correlations explored the variability of these responses throughout the film. Both groups showed similar brain and pupil responses to ToM events and provided comparable descriptions of the characters' mental states. However, participants with autism exhibited significantly stronger correlations in their responses across the film's social narrative, indicating reduced inter-individual variability. This distinct pattern emerged well before any ToM events and involved brain regions beyond the ToM network. Our findings provide functional evidence of spontaneous mentalizing in autism, demonstrating this capacity in a context affording but not requiring mentalizing. Rather than responses to ToM events, a novel neurocognitive signature - inter-individual variability in brain and pupil responses to evolving social narratives - differentiated neurotypical individuals from those with autism. These results suggest that idiosyncratic narrative processing in unstructured settings, a common element of everyday social interactions, may offer a more sensitive scenario for understanding the autistic mind.

\thispagestyle{empty}
