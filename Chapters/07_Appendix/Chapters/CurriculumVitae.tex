\section{Biography}
Tim van Mourik was born on September 27th 1990 in Leiden, the Netherlands. After graduating from the Stedelijk Gymnasium Leiden, he went on to pursue a Bachelor's degree in the exact sciences at the Roosevelt Academy (currently: University College Roosevelt). After this, he enrolled in the Master Computer Animation and Visual Effects. Over the course of this program, he started to develop an interest in medical image processing and completed the program with an internship at the Donders Centre for Cognitive Neuroimaging under the supervision of prof. David Norris. As MRI scanners operate on physical and mathematical principles, subsequently producing three-dimensional images, this perfectly combined Tim's expertises and interests. He continued to work as a research assistant to be in initiated in the world of neuroimaging and laminar analysis. In February 2014, he started a four-year PhD project to investigate and better understand the cortical layers of the brain. During this time he developed methods for better analysing laminar fMRI data and under the supervision of Dr. Janneke Jehee, he conducted a study on the layer specificity of spatial attention. All tools, code, and data for his published work can be found only and Tim is a strong proponent of a more open way of doing science, data and code sharing, and open source development. In this spirit, he programmed worfklow software, Porcupine, that automatically and more transparently creates analysis code. He is currently contuining this line of work by creating an online platform for easier data sharing, GiraffeTools. It is unknown where he acquired his peculiar taste for silly acronyms. For these efforts towards open science he was elected to be the chair of the Open Science Room at OHBM 2019, Rome. 
