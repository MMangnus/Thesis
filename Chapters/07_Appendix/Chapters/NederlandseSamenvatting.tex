\section{Nederlandse samenvatting}
In dit proefschrift zoomen we zo ver mogelijk in op het brein, en proberen we een glimp op te vangen van de processen die daar plaatsvinden. Door iedere paar seconden een driedimensionale afbeelding te maken van het brein kunnen we een indruk krijgen wat er zich daar afspeelt. Op een grove schaal kunnen we zien welke breingebieden actief worden bij verschillende taken: als het licht aangaat, wordt de achterkant van je brein actief (de \emph{visuele cortex}), en voornamelijk de linker zijkant van het brein (je taalcentrum) wordt actief als je deze tekst leest. Maar wat betekent het dat die gebieden actief worden? Wat doen ze dan precies? Op een MRI-scanner kunnen we voornamelijk zien dat gebieden meer of minder zuurstof verbruiken. Dat vertelt echter niet zo gek veel over het proces wat zich daar afspeelt. Om daar iets beter achter te komen, moeten we nog verder kijken. Dankzij anatomische ontledingen van het brein weten we dat de buitenste schil van het brein, de grijze stof, uit verschillende laagjes bestaat. Sommige lagen ontvangen informatie en sommige lagen sturen het door naar de andere breingebieden. Het doel van dit proefschrift is de functie van verschillende lagen te laten zien op basis van MRI-scans.

Dat is echter niet makkelijk: de laagjes zijn zo klein dat de resolutie van de MRI-scanner maar ternauwernood goed genoeg is. En daar komt nog bij dat de plaatjes die uit een MRI-scanner komen zelf ook licht verschoven zijn. Vervolgens moet je erachter zien te komen hoe, in een kronkelend brein, de laagjes precies verdeeld zijn over alle volume-pixels (voxels) van het driedimensionale plaatje. En verder wilden we dit niet gewoon \'e\'en keer uitvoeren, maar zorgen dat iedereen dit type onderzoek voortaan ook kan doen.

% hoofdstuk 2
In \textbf{hoofdstuk 2} zijn we begonnen met het eerste probleem: binnen de afbeeldingen die uit de MRI-scanner komen, liggen de verschillende lagen niet exact op de goede plek. Dit heeft te maken met de manier waarop een MRI-scanner werkt: in het midden van scanner wordt een groot magneetveld gecre\"eerd wat cruciaal is voor het maken van scans. Echter, als het magneetveld niet exact homogeen is, vertaalt dit zich in de scan als kleine (lokale) verplaatsingen. Omdat het nou juist cruciaal is tot op het kleinste niveau op de goede plek te meten als je de verschillende lagen wil meten, is dit een groot probleem. Daarom hebben we een methode bedacht om de scan `recht te trekken' en weer terug op de goede plek te leggen. Op basis van beeldanalyse kijken we heel precies waar de grenzen van de witte en grijze stof zitten, ten opzichte van een niet verplaatste referentie scan. We laten vervolgens op verschillende data sets zien dat de methode daadwerkelijk een veel preciezer beeld geeft over de locatie van de verschillende lagen. 

% hoofdstuk 3
Hiermee is de locatie van de lagen met hoge precisie bekend. Het hele hersenoppervlak ligt echter nog steeds gekronkeld in het volume en de verschillende lagen zijn zo klein dat ze nog niet duidelijk te zien zijn. In \textbf{hoofdstuk 3} beschrijven we een nieuwe methode op basis waarvan de signalen uit verschillende lagen beter geschat kunnen worden. Dit doen we door middel van een wiskundig model dat beschrijft hoe de laagjes verdeeld zijn over het volume. Aan de hand van dit model kunnen we preciezer dan voorheen de signalen uit de lagen halen. Op basis van drie verschillende datasets laten we zien hoe deze nieuwe methode presteert. In een simulatie van een MRI-volume blijkt het veel beter te werken dan bestaande methode. Dit is echter niet zo duidelijk in data van echte scans. De reden hiervoor is dat onze methode op goede data inderdaad een betere schatting kan maken, maar wel gevoeliger is voor ruis in de data. Dat maakt het moeilijk goed zicht te krijgen op het verschil tussen de methodes op standaard MRI-scans.

% hoofdstuk 4
Na deze twee methodologische studies om de weg vrij te maken voor laag-specifieke analyse, hebben we in \textbf{hoofdstuk 4} geprobeerd uit te zoeken of de lagen inderdaad verschillende activatie tonen bij verschillende processen. Hiervoor hebben we een aandachtstaak gebruikt. We vroegen mensen hun aandacht links of rechts te focussen en af en toe verscheen er een stimulus (een zwart-wit gestreept plaatje) op het scherm. Zo wilden we onderzoeken of het effect van aandacht in andere lagen te zien valt dan het effect van het zien van een stimulus. Op brein niveau was er wel degelijk een effect te merken, maar op laagniveau zagen we geen verschil. Dit is dus een nul-resultaat met betrekking tot de signalen uit de verschillende lagen. Het is moeilijk te zeggen wat hier precies de reden van is: of het effect niet bestaat, of het effect niet sterk genoeg was, of dat de methodes niet goed genoeg waren om het eruit te halen (of een combinatie van factoren).

% hoofdstuk 5
\textbf{Hoofdstuk 5} gaat over een applicatie die we ontwikkeld hebben om het makkelijker te maken om MRI data te analyseren, Porcupine. Analyses bestaan doorgaans uit lange scripts die een aaneenschakeling beschrijven van verschillende bewerkingen die uitgevoerd moeten worden op de gemeten data. Onderzoekers schrijven deze analyse code vaak zelf, maar deze scripts zijn vaak moeilijk te interpreteren. In plaats hiervan kan een analyse in Porcupine visueel geprogrammeerd worden door blokjes aan elkaar te verbinden. Dit representeert de volgorde van de analysestappen en laat zo op een grafische manier zien hoe de `workflow' eruit ziet. Porcupine genereert vervolgens de analyse code die direct uitgevoerd kan worden. Omdat dit een inzichtelijke weergave biedt die makkelijker deelbaar is en makkelijker valt aan te passen dan huidige analyse scripts, zorgt dit voor een reproduceerbaardere wetenschap.

% Concluding remarks
Hiermee maakt het werk in dit proefschrift de weg vrij om de lagen van de cortex meer routinematig te analyseren. Alhoewel we in dit onderzoek geen effecten gevonden hebben, lijken andere studies (die dezelfde methodes hebben gebruikt) wel resultaten op te leveren. Het is dus nog onduidelijk wat voor informatie er precies te vinden is als we met die precisie in het brein proberen te kijken. Laag-specifieke brein analyse in functionele MRI staat hiermee dus nog in de kinderschoenen, en onze methodologische ontwikkelingen staan hieraan ten grondslag.




