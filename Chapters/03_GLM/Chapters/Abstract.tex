There is converging evidence that distinct neuronal processes leave distinguishable footprints in the laminar BOLD response. However, even though the achievable spatial resolution in functional MRI has much improved over the years, it is still challenging to separate signals arising from different cortical layers. In this work, we propose a new method to extract laminar signals.
We use a spatial General Linear Model in combination with the equivolume principle of cortical layers to unmix laminar signals instead of interpolating through and integrating over a cortical area: thus reducing partial volume effects. Not only do we provide a mathematical framework for extracting laminar signals with a spatial GLM, we also illustrate that the best case scenarios of existing methods can be seen as special cases within the same framework. 
By means of simulation, we show that this approach has a sharper point spread function, providing better signal localisation. 
We further assess the partial volume contamination in cortical profiles from high resolution human ex vivo and in vivo structural data, and provide a full account of the benefits and potential caveats. We eschew here any attempt to validate the spatial GLM on the basis of fMRI data as a generally accepted ground-truth pattern of laminar activation does not currently exist.
This approach is flexible in terms of the number of layers and their respective thickness, and naturally integrates spatial regularisation along the cortex, while preserving laminar specificity. Care must be taken, however, as this procedure of unmixing is susceptible to sources of noise in the data or inaccuracies in the laminar segmentation.
