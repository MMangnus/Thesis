\chapter*{Curriculum Vitae}
\label{ch:cv}
\chaptermark{Curriculum Vitae}
\addcontentsline{toc}{chapter}{Curriculum Vitae}

Margot Mangnus was born on September 12 1995 in Terneuzen. She obtained a Bachelor's degree in 2017 in Linguistics at Radboud University Nijmegen with a focus on electives in Neuroscience. Her Bachelor's thesis involved investigating the effect of noise-induced hearing loss on the perception of Dutch consonants in different types of background noise, supervised by Dr.~Odette Scharenborg. 

She obtained a Master's degree in Cognitive Neuroscience (cum laude) in 2018, specializing in the track Language \& Communication. Her Master's thesis and internship were under the supervision of Dr.~Nikki Janssen, Dr.~Vit\'{o}ria Piai and Prof.~Dr.~Ardi Roelofs. For this thesis, she investigated the role of the ventral white matter pathway in language production in healthy individuals and patients with Primary Progressive Aphasia. 

In 2019, she joined the Neurobiology of Language lab, led by Prof.~Dr.~Peter Hagoort, and the Mutual Understanding lab, led by Dr.~Arjen Stolk and Dr.~Jana Ba\v{s}n\'{a}kov\'{a}. During this time, she supervised a Master's thesis project together with Dr.~Ba\v{s}n\'{a}kov\'{a}. She was also an active member of the PhD Council for 3.5 years, involved in organizing events, workshops and initiatives to improve policy and mental health of Donders PhD students. The current thesis consists of a selection of her PhD work during 2019 and 2023.