
\section*{Thesis outline}
This thesis covers two major problems in laminar fMRI that needed to be solved before an experimental study could be conducted, and will reflect on building an fMRI pipeline, laminar or otherwise. In Chapter~\ref{ch:registration}, we will discuss a new way of coregistering an anatomical scan with a functional scan, when the latter is non-linearly distorted. We will explain the details of the distortion correction technique, show its performance, and freely provide the code and data online. Chapter~\ref{ch:glm} describes a novel way of extracting the laminar signal from data. We show this method’s performance on a range of data, from a simulated fMRI model to post-mortem data, to \emph{in vivo} data from a set of subjects. Having overcome several of the most challenging aspects of laminar analysis, in Chapter~\ref{ch:attention} we then proceed to a laminar experiment. In a visual attention experiment, we investigate the laminar response. In Chapter~\ref{ch:porcupine}, we further develop a new tool to more easily build an fMRI analysis pipeline, to more reproducibly conduct science, and to easily share analysis pipelines with others. Finally, these results will be put in a broader perspective in the Discussion, Chapter~\ref{ch:discussion}.