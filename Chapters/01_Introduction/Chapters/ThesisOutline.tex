
\section*{Thesis outline}
This thesis covers two major problems in laminar fMRI that needed to be resolved before an experimental study could be conducted, and will reflect on building an fMRI pipeline, laminar or otherwise. In Chapter~\ref{ch:registration}, we will discuss a new way of coregistering an anatomical scan with a functional scan, when the latter is non-linearly distorted. We explain the details of the distortion correction technique, show its performance, and freely provide the code and data online. Chapter\ref{ch:glm} describes a novel way of extracting laminar signal from data. We show its performance on a multitude of data, from a simulated fMRI model to post mortem data, to in vivo data from a set of subjects. Having overcome several of the most challenging aspects of laminar analysis, we then proceed to a laminar experiment in Chapter~\ref{ch:attention}. In a visual attention experiment, we investigate the laminar response. We further developped a new tool to more easily build fMRI analysis pipeline, to more reprodicibly conduct science, and to easily share analysis pipelines with others in Chapter~\ref{ch:porcupine}. Finally, these results will be put in a broader perspective in the Discussion, Chapter~\ref{ch:discussion}.