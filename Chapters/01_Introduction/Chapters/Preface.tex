\section*{Preface}
``How does that work?'' That may be the fundamental question of the natural sciences. 
Over the centuries science has discovered ever smaller particles from molecules, to atoms, to quarks. On the other side of the spectrum, we understand more and more about our solar system, galaxy, and entire universe. And somewhere in between, a set of awkwardly arranged molecules forms you: a living breathing and thinking human being.
Now this is certainly not the only thing in the universe of which it is interesting to know how it works, but there is something unique about this level: the fact that it feels like we are not merely at the whim of natural forces bouncing us around, but that we can exert control on our movement; the fact that it feels like anything at all. There has to be a way that these feelings are instantiated by our molecules, our cells, and by our brain.
The field of neuroscience tries to get a better grasp on this `from molecule to man' approach. In this thesis, we will zoom in on a small piece of this puzzle: can we better understand communication between different brain regions by looking at MRI brain scans at even closer detail.