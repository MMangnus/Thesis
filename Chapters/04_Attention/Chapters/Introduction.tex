\section{Introduction}
Directing visual attention to a location in the visual field typically improves behavioral sensitivity to stimuli presented at that location \cite{Posner1980,Lee1997,Yeshurun1998,Carrasco2004,Baldassi2005,Ling2009}. It is well known that these attentional benefits in behavior are accompanied by increases in BOLD response in early visual areas (e.g. \cite{Brefczynski1999,Gandhi1999,Kastner1999}), but how top-down processes modulate cortical responses at the laminar level remains unknown.

It is known from anatomical studies that the human cerebral cortex can be subdivided into histological layers with different cell types. The cytoarchitectonic structure varies across the brain and forms the basis of the Brodmann atlas, but almost all brain areas have six different histological layers \cite{Brodmann1909}. Although the precise function of each cortical layer remains unclear, their connectivity profile suggests a division in terms of bottom-up and top-down processing \cite{Felleman1991}. Specifically, Layer IV and to a lesser extent Layer V/VI are commonly associated with receiving feedforward drive from Layer III of lower cortical areas or from the thalamus \cite{Jones1998,Constantinople2013}. Layers I-II and VI, in contrast, are typically implicated in receiving downward information flow (feedback), which often originates from layer V \cite{Alitto2003}. Interestingly, this bottom-up versus top-down connectivity profile of each of the layers is to some degree paralleled in functional data. That is, from neurophysiological and neuroimaging work, it is known that various visual stimuli and tasks can exert differential effects on the various layers \cite{Maier2010,Xing2012,Self2013,VelezFort2014, OHerron2016}. Intracranial work in monkeys, for instance, shows that for selective attention and working memory (two functions that are commonly associated with top-down processes), current source density is increased in deep and superficial compared to middle layers in primary visual cortex \cite{VanKerkoerle2017}. Similar layer specific patterns have been shown in animal functional MRI. For instance, whisker stimulation led to an increase in BOLD response in Layer IV of rat barrel cortex, before such an enhancement was observed in any of the other layers, suggesting that layer IV was the first to receive feed forward drive from lower-level areas \cite{Yu2014}. In contrast, subsequent corticocortical connections in the same task appeared to activate Layers II-III and V in the motor cortex and contralateral barrel cortex before this affected any of the other layers, suggesting that these layers were the first to receive feedback signals. To what extent these results generalize to human cortex, however, remains to be investigated.

Recent advancements in fMRI have made it possible to also investigate the functional role of cortical layers in humans (e.g. \cite{Polimeni2010,Maass2014,Kok2016}). The human in vivo resolution with fMRI has increased to submillimetre voxel size. The thickness of the cerebral cortex varies between 1 and 4.5 millimetres \cite{Zilles1990,Fischl2000}, suggesting sufficient resolution to characterise activity across the individual layers. Indeed, some evidence suggests that human cortical activation can be measured with fMRI on a layer specific basis \cite{ Kok2016, Muckli2015}. For example, illusory contours, which are commonly associated with top-down processes, appear to activate the deep layers more than any of the other layers In area V1 \cite{Kok2016}, and some findings suggest that also specific activation of the middle layers can be measured with fMRI in human primary visual cortex \cite{Koopmans2010}.

While some neurophysiological evidence suggests a differential involvement of the cortical layers in top-down attention \cite{Nandy2017}, the effects of attention on the different layers in human visual cortex has remained unclear. Here, we examine with fMRI the potential influence of spatial attention on BOLD activity in the deep, middle and superficial layers in human visual areas V1, V2, and V3. Participants directed their attention to a cued location, and performed an attention-demanding task using an orientation stimulus that was shown at this location, while an unattended grating appeared at a different location of equal eccentricity. On some of the trials, subjects directed their attention to the cued location in anticipation of the stimulus, but no stimulus appeared at this location. Interestingly, although we observed a reliable increase of the overall BOLD response with attention across all layers, both with and without a stimulus present, we observed no differences in activation level between the layers due to attention. We provide several reasons for these surprising findings in the Discussion.
