
%%%%%%%%%%%%%%%%%%%%%%
%\usepackage[sc]{mathpazo} % Use the Palatino font
\usepackage[T1]{fontenc} % Use 8-bit encoding that has 256 glyphs
%\usepackage[default,osfigures,scale=0.95]{opensans}
\renewcommand{\familydefault}{\sfdefault}
\usepackage{microtype} % micro-type adjustments, prevents a lot of bad boxes
\usepackage{nameref}
\usepackage[hidelinks, bookmarksnumbered=true, bookmarksdepth=1]{hyperref} % hidelinks disables
\usepackage{url}
\usepackage{float} 		% adds specific locations with the [H] (e.g. \begin{table}[H])
\usepackage{titlesec} 	% Allows customization of titles
\usepackage{tocloft}
\usepackage{xcolor} 	%use colours, e.g. 'gray' in chapter title page
\usepackage{lmodern}
\usepackage{listings}
%\usepackage{geometry}
\usepackage[dutch,english]{babel} % language and hyphenation (put main language last)
\usepackage{graphicx}	% inclusion of graphics
\usepackage[export]{adjustbox}% allows centering figures that are wider than text 
\usepackage{mathtools}	% math tools
\usepackage{textcomp}	% nice math symbols in text
\usepackage{etoolbox}	% provides things like convenient if-else-statements
\usepackage{enumitem}	% for better enumeration
\usepackage{multicol}	% two-col the reference list
\usepackage{etoolbox}	% style reference list 
\usepackage{relsize}	% style reference list
\usepackage{appendix}

\linespread{1.2} % Bigger spacing for the reading committee

% Used for blank pages
\usepackage{afterpage}
\newcommand\blankpage{
    \null
    \thispagestyle{empty}
    \newpage}

%geometry and layout
\pagestyle{plain} % center all page numbering
\usepackage[
	inner=30mm,
	outer=25mm,
	bottom=21.5mm,
	footskip=13.5mm,
]{geometry}
\textwidth = 115mm
\voffset = -14mm
%\topmargin = 14mm
\headsep 14mm

\usepackage[finalnew]{Stylesheets/trackchanges}
%%%%%%%%%%%%%%%%%%%%%%

\usepackage{silence}
\WarningFilter*{memoir}{You are using the caption package with the memoir class}
\WarningFilter{latex}{Text page}

% DIN B5 format
\setstocksize{240mm}{170mm}
\settrimmedsize{\stockheight}{\stockwidth}{*}
%\checkandfixlayout
%\usepackage{polyglossia}
%\setdefaultlanguage[variant=british]{english}
\usepackage[format=hang,font=footnotesize,labelfont={sc,bf}]{caption}% styling of figure/table captions


%\newtoggle{manus}
%\togglefalse{manus}

%%%%%%%%%%%%%%%%%%%%%%%%%%%%%%%%%%%%%%%%%%%%%%%%%%%%%%%%%%%%%%%%%%%%%%%%%%%%%%%%
% Macros, environments, etc.
%%%%%%%%%%%%%%%%%%%%%%%%%%%%%%%%%%%%%%%%%%%%%%%%%%%%%%%%%%%%%%%%%%%%%%%%%%%%%%%%

% memoir class options
\nouppercaseheads
%\setsecnumdepth{subsection}

% margins
%\setulmarginsandblock{3cm}{3.7cm}{*}

%\iftoggle{manus}{%
%	\setlrmarginsandblock{3cm}{3cm}{*}
%}{%
%	\setlrmarginsandblock{2cm}{3cm}{*}
%}

\setheadfoot{1cm}{1.5cm}
\setheaderspaces{*}{0.8cm}{*}
%\checkandfixthelayout

% contents of headings
\copypagestyle{myheadings}{headings}

% patch foot rule to ensure outer margin aligned stripe
\makeatletter
\patchcmd{\makefootrule}
  {\hrule\@width #2\@height #3 }
  {\rule{#2}{#3}}
  {}
  {}
\makeatother

\makeheadfootruleprefix{myheadings}{}{%
  \checkoddpage\ifoddpage\hspace*{\dimexpr\textwidth-8mm\relax}\fi}
\makeheadrule{myheadings}{\textwidth}{\normalrulethickness}

\makepsmarks{myheadings}{%
	\createmark{chapter}{both}{nonumber}{}{\space}
}

%\iftoggle{manus}{%
%	\makeevenhead{myheadings}{\footnotesize\textsc{Manuscript Doctoral Thesis Tim van Mourik}}{}{\footnotesize\thepage}
%	\makeoddhead{myheadings}{\footnotesize\textsc{\MakeLowercase{\leftmark}}}{}{\footnotesize\thepage}
%}{%

% everything in header
	%\makeevenhead{myheadings}{\footnotesize\thepage}{}{\footnotesize\textsc{Manuscript Doctoral Thesis Tim van Mourik}}
	%\makeoddhead{myheadings}{\footnotesize\textsc{\MakeLowercase{\leftmark}}}{}{\footnotesize\thepage}
	
% page number in footer
	
%	\makeevenhead{myheadings}{\footnotesize\textsc{chapter \thechapter}}{}{}
%	\makeoddhead{myheadings}{}{}{\footnotesize\textsc{\MakeLowercase{\leftmark}}}
%	\makeevenfoot{myheadings}{\footnotesize\thepage}{}{}
%	\makeoddfoot{myheadings}{}{}{\footnotesize\thepage}
%}

% headings for back matter
\copypagestyle{backheadings}{myheadings}

% label for subcaption (i.e. A)
\newcommand*{\subcap}[1]{ \textbf{\textsc{(#1)}} }

% reference Figure
\newcommand*{\figref}[2]{Figure \ref{fig:ch\thechapter:fig#1}\textsc{#2}}
\newcommand*{\figrefplain}[2]{\ref{fig:ch\thechapter:fig#1}\textsc{#2}}

% more stuff used within model paper
\newcommand{\unit}[1]{\ensuremath{\, \mathrm{#1}}}
\newcommand{\spps}{\ensuremath{\, \mathrm{sp}/\mathrm{s}}}
\newcommand{\cv}{\ensuremath{\mathit{CV}}} % reduces spacing between the c and the v

% Convenient super- and subscripting in text environment
\newcommand{\super}[1]{\ensuremath{^{\textrm{#1}}}}
\newcommand{\sub}[1]{\ensuremath{_{\textrm{#1}}}}

\setlength{\bibitemsep}{.5pt}

% only show chapters in ToC, no sections
\setcounter{tocdepth}{0}

% used to align figures to inner margin in book, or centered for manuscript
%\iftoggle{manus}{%
%	\newcommand{\bigfigalign}{center}
%}{%
%	\newcommand{\bigfigalign}{inner}
%}

% typeset subsubsection headings inline (used in model chapter)
\makeatletter
\def\subsubsection{\@startsection{subsubsection}{3}{1.0em}{0.5em}{-1.5em}{\normalsize\itshape}}
\makeatother

% don't care about slightly bad boxes
\hfuzz=2pt
\vbadness=9000
%\hbadness=9000

